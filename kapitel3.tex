\chapter{Grundlagen f�r die folgenden Kapitel}
F�r das Verst�ndnis der nachfolgenden Kapitel werden kurz wichtige Verfahren
benannt und Techniken beschrieben, die Anwendung finden. Weiterhin werden
Hinweise zu weiterf�hrender Literatur gegeben.

\section{MDA}

	In der Vergangenheit der Softwareentwicklung zu beobachten war, dass wenn zu
	erstellende Software-Systeme auf ein nicht mehr zu beherrschendes Ma�
	angewachsen waren, die L�sung darin bestand auf eine h�here semantische Ebene
	zu gehen (Abstraktionssprung der Programmiersprache). Ausgehend von der
	Tatsache, dass Maschinensprachen �ber Assembler, Prozedurale Sprachen durch
	Objektorientierte Sprachen abgel�st wurden l�sst sich schlussfolgern, dass die
	Zukunft des Software-Engineerings in den Modellierungssprachen liegt.
	[Gruhn06], S. 14ff
	
	%TODO: Grafik aus Buch S. 15
	
	\subsection{Begriffskl�rung}
      	\begin{quote}
      	``MDA ist ein Standard der Object Management Group (kurz OMG). Die OMG
      	selbst wurde 1989 gegr�ndet und ist ein offenes Konsortium aus Firmen
      	weltweit. Die OMG erstellt herstellerneutrale Spezifikationen zur
      	Verbesserung der Interoperatibilit�t und Portierbarkeit von
      	Softwaresystemen und ist traditionell eine Plattform f�r Middleware- und
      	Tool-Hersteller zur Synchronisation und Standardisierung ihrer
      	Bet�tigungsfelder.'' [Stahl07], Seite 377
      	\end{quote}

	\subsection{Arbeitsweise}
		MDA arbeitet mit mehreren Schichten:
		
		\begin{itemize}
          \item Computation Independent Model (CIM)\\
          		Das CIM liefert eine Sicht auf das Gesamtsystem unabh�ngig davon,
          		wie es implementiert werden soll. Es enth�lt die Anforderungen des
          		Systems an die Umwelt.
          
          \item Platform Independent Model (PIM)\\
          		Das PIM beschreibt die formale Struktur und die Funktionalit�t des
          		Systems.
          		
          \item Platform Specific Model (PSM)\\
          		Durch Anreicherung des PIM mit plattform-abh�ngigen Informationen
          		entsteht das PSM. 
          		
          \item Code\\   %TODO: check: richtig?
          		Aus dem PSM wird der Quellcode f�r die Zielplattform generiert.
				Meist entsteht dabei noch kein ausf�hrbarer Code, sondern lediglich
				ein Grundger�st, das f�r die weitere h�ndische Implementierung genutzt wird.  
        \end{itemize}
        
        Transformationen bilden die Grundlage f�r die �berf�hrung von einer
        Schicht in eine andere. Dabei unterscheidet man zwischen:
        
        \begin{itemize}
          \item Modell-zu-Modell Transformation
          \item Modell-zu-Code Transformation
        \end{itemize}
        
        Kernidee der MDA ist es also vom unabh�ngigen Modell �ber
        Transformationen bis hin zum eigentlichen Programmcode zu gelangen. 
        
        Ziele der MDA:
        
        \begin{itemize}
          \item Konservierung der Fachlichkeit
          \item Portierbarkeit
          \item Systemintegration und Interoperabilit�t
          \item Effiziente Softwareentwicklung
          \item Dom�nen-Orientierung
        \end{itemize} [Gruhn06], S. 21 ff
        
        Mehr Informationen zur Model Driven Architecture findet man auf der
        Webseite der OMG\footnote{http://www.omg.org} oder als Literatur
        [Gruhn06], in diesen detailiert auf die Softwareerstellung mit Hilfe
        von MDA (auch an Beispielen) eingegangen wird.
        
 %TODO       
    \section{ISO/OSI-Modell}
    
    OSI Model
	Data unit 	Layer 	Function
Host
layers 	Data 	7. Application 	Network process to application
6. Presentation 	Data representation, encryption and decryption, convert machine dependent data to machine independent data
5. Session 	Interhost communication
Segments 	4. Transport 	End-to-end connections and reliability,flow control
Media
layers 	Packet 	3. Network 	Path determination and logical addressing
Frame 	2. Data Link 	Physical addressing
Bit 	1. Physical 	Media, signal and binary transmission

	\begin{table}%[h]
    	\centering \leavevmode %Tabelle zentrieren
        \caption{OSI-Modell}
        \label{tab:iso-osi}
        %\addtocounter{footnote}{+1}
        \begin{tabular}{||c||c||c||}
        	\hline \hline
        	\multicolumn{3}{||c||}{OSI Model}\\
        	\hline \hline
       		Data Unit(Einheit) & Layer & Funktion\\
       		\hline \hline
        \end{tabular}
	\end{table}
			%\addtocounter{footnote} 
			%\footnotetext{[CT10], S. 101}
    
    
    \subsection{Layer}
    
    \subsection{TCP/IP Header}
    
    %TODO: Bild von Header einf�gen. evtl. einzelne begriffe kl�ren. oder
    % kurzen text dazu formulieren
        