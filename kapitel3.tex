\chapter{Grundlagen f�r die folgenden Kapitel}
F�r das Verst�ndnis der nachfolgenden Kapitel werden kurz wichtige Verfahren
benannt und Techniken beschrieben, die Anwendung finden. Weiterhin werden
Hinweise zu weiterf�hrender Literatur gegeben.

\section{MDA}

	In der Vergangenheit der Softwareentwicklung zu beobachten war, dass wenn zu
	erstellende Software-Systeme auf ein nicht mehr zu beherrschendes Ma�
	angewachsen waren, die L�sung darin bestand auf eine h�here semantische Ebene
	zu gehen (Abstraktionssprung der Programmiersprache). Ausgehend von der
	Tatsache, dass Maschinensprachen �ber Assembler, Prozedurale Sprachen durch
	Objektorientierte Sprachen abgel�st wurden l�sst sich schlussfolgern, dass die
	Zukunft des Software-Engineerings in den Modellierungssprachen liegt. [Gruhn06]
	
	%TODO: Grafik aus Buch S. 15
	
	\subsection{Begriffskl�rung}
      	\begin{quote}
      	``MDA ist ein Standard der Object Management Group (kurz OMG). Die OMG
      	selbst wurde 1989 gegr�ndet und ist ein offenes Konsortium aus Firmen
      	weltweit. Die OMG erstellt herstellerneutrale Spezifikationen zur
      	Verbesserung der Interoperatibilit�t und Portierbarkeit von
      	Softwaresystemen und ist traditionell eine Plattform f�r Middleware- und
      	Tool-Hersteller zur Synchronisation und Standardisierung ihrer
      	Bet�tigungsfelder.'' [Stahl07], Seite 377
      	\end{quote}

	\subsection{Arbeitsweise}
		MDA arbeitet mit mehreren Schichten:
		
		\begin{itemize}
          \item Computation Independent Model (CIM)
          \item Platform Independent Model (PIM)
          \item Platform Specific Model (PSM)
          \item Platform Description Model (PDM)   %TODO: check: richtig?
        \end{itemize}
        
        Transformationen bilden die Grundlage f�r die �berf�hrung von einer
        Schicht in eine andere. Dabei unterscheidet man zwischen:
        
        \begin{itemize}
          \item Modell-zu-Modell Transformation
          \item Modell-zu-Code Transformation
        \end{itemize}
        
        Ziel der MDA ist es also vom unabh�ngigen Modell �ber Transformationen
        bis hin zum eigentlichen Programmcode zu gelangen. 
        
        Mehr Informationen zur Model Driven Architecture findet man auf der
        Webseite der OMG\footnote{http://www.omg.org} oder als Literatur
        [Gruhn06], in diesen detailiert auf die Softwareerstellung mit Hilfe
        von MDA (auch an Beispielen) eingegangen wird.
        
 %TODO       
    \section{ISO/OSI Modell}
    
    \subsection{Layer}
    
    \subsection{TCP/IP Header}
    
    %TODO: Bild von Header einf�gen. evtl. einzelne begriffe kl�ren. oder
    % kurzen text dazu formulieren
        