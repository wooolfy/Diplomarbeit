% Hyperref Einstellungen
\hypersetup{
 pdfauthor={Georg Wolf},
 pdftitle={Erstellung von Apps zum Einsatz auf Handys und Mobil-PCs zur
 Kommunikation mit Maschinen und Linien}, pdfsubject={Diplomarbeit, FH
 Gummersbach}, pdfkeywords={Diplomarbeit, Smartphone Apps, Android, iOS},
 backref=true,
 pagebackref=true,
 hyperfigures,
 hyperindex,
 colorlinks,
 bookmarksopen,
 bookmarksopenlevel=1,
 bookmarksnumbered,
 pageanchor,
 plainpages=false,
 % Hyperlink Farben:
 urlcolor=black,
 pagecolor=black,
 menucolor=black,
 citecolor=black,
 anchorcolor=black,
 filecolor=black,
 linkcolor=black,
 colorlinks=true, % Links einfaerben oder umranden
}

\setlength{\parindent}{0pt}	% Einr�cktiefe bei Abs�tzen einstellen
\setlength{\parskip}{1.5ex}	% Absatzabstand einstellen

% Mit folgenden Werten kann man die Silbentrennung einstellen (0..10000).
%\hyphenpenalty=10000 
%\exhyphenpenalty=10000
%\hyphenpenalty=10 \exhyphenpenalty=10

\setcounter{secnumdepth}{3}	% Hier stellt man ein bis zu welcher Ordnung �berschriften nummeriert werden (Beginnt bei 0 (Kapitel) und geht bis 5 (Subparagraph)
\setcounter{tocdepth}{3}	% Hier stellt man ein bis zu welcher Ordnung �berschriften ins Inhaltsverzeichnis �bernommen werden (ebenfalls 0 - 5)
%

\setlongtables

\graphicspath{{images/}{/global/images/}{anhang/}}	% Pfade f�r Grafiken angeben

\makeindex

% Anzahl und max. Seitenanteil von Abbildungen und Tabellen einstellen
\setcounter{topnumber}{5}
\setcounter{bottomnumber}{5}
\renewcommand{\bottomfraction}{.9}
\renewcommand{\textfloatsep}{3ex}
\renewcommand{\floatsep}{3ex}

% Nicht empfehlende Einstellungen:
%\sloppy    % Schlampige Absatzformatierung (wie M$ Word)
%\usepackage{wordlike}    % Simuliert M$ Word 

