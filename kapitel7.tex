\chapter{Implementierung und Test}%Softwareerstellung}
		\section{Wahl des Betriebssystems}
		\begin{itemize}
          \item tja\ldots Entweder nur eins ausw�hlen oder Android und iOS und
          Win7 Mobile (oder Webinterface\ldots) -> ZU VIEL ARBEIT?
          \item Wahrscheinlich muss die Entwicklung f�r iOS ausgeschlossen
          werden, weil die entsprechende Entwicklungshardware fehlt. ODER:
          JAILBREAK -> Eclipse -> TESTEN LIVE AUF DEM GER�T OHNE SIMULATOR
          (zumindest theoretisch auf diese M�glichkeit eingehen).
          \item gibt es evtl. freie iphone emulatoren? ->derzeitige Recherche->
          NEIN
          \item Entwicklungsumgebung f�r Android und Win7 Mobile ist kostenlos.
          \item Annahne: Wahrscheinlich f�llt die Wahl auf Android, da hier
          auch direkt auf einem Ger�t getestet werden kann und nicht nur auf
          dem Simulator. Zu demonstrationszwecken ist dies wohl sinnvoller. 
        \end{itemize}
        \section{Codegenerierung}
        
        \subsection{Vorhandene Tools auf dem Markt}

      Verzichtet werden soll auf die Verwendung kommerziell-, propriet�rer
      Software zugunsten offener Standards und frei verf�gbarer
      Open-Source Technologien. Dabei sind auch �berlegungen zu den
      unterschiedlichen Lizenzen anzustellen.
      
      Deswegen werden im Folgenden erst grafische Tools zur Modellierung der
      Diagramme untersucht und anschlie�end Transformationstools, die f�r die
      �berf�hrung der einzelnen Modelle und schliesslich f�r den entstehenden
      Quellcode verantwortlich sind.
      
 	  \subsubsection{Visual Paradigm}
      
      Visual Paradigm ist sehr leicht und intuitiv zu erlernen. Die meisten
      Funktionen findet man auf anhieb, ohne auch nur einen Blick in die
      Dokumentation werfen zu m�ssen. Allerdings macht das arbeiten mit Visual
      Paradigm in Kombination mit einem Transformationstool nicht
      viel Sinn, da beim Export der Diagramme in XML Files Informationen verloren
      gehen. Beim Import in Acceleo beispielsweise fehlen die Namen bzw.
      Bezeichnungen der Diagramme so wie Zusammenh�nge der einzelnen
      Diagrammarten, die jedoch vorher definiert wurden. Damit ist Visual
      Paradigm f�r den Zweck der Codegenerierung relativ unbrauchbar.

      \subsubsection{MagicDraw}

      Ein erster Versuch ein einfaches Klassendiagramm zu implementieren und in
      Code umzuwandeln klappte erstaunlicherweise recht gut. Allerdings ist die
      Bedienung von MagicDraw wesentlich komplexer und un�bersichtlicher als bei
      Visual Paradigm. Die Einarbeitungszeit ist bei dieser Komplexit�t
      exponentiell gr��er. Daher verz�gerte sich die Erstellung der Diagramme
      sehr. Was in Visual Paradigm in Stunden machbar war, dauerte hier
      einen ganzen Tag. Schlechte Navigation, un�bersichtliche Men�s und
      unzureichende Dokumentation zur Software, aber daf�r volle
      Kompatibilit�t zum XML Standard.
      
      \subsubsection{Acceleo}
      
      Einfache Klassendiagramme in Code umzuwandeln ist mit dem Tool kein
      Problem. Allerdings ist die Dokumentation in Bezug auf Diagrammarten
      ziemlich unvollst�ndig. Das Durchlaufen von Aktivit�tsdiagrammen ist
      prinzipiell anhand der Activities m�glich. Allerdings konnte ich nicht
      herausfinden, wie man diese in der richtigen Reihenfolge darstellt.
      Acceleo wirft die Ergebnisse einfach so durcheinander, in der Reihenfolge
      wie sie im XML File des UML's generiert wurden. Aus einem Kontrollfluss
      heraus etwas zu erzeugen stellt sich also als ziemlich unm�glich dar.
      
      \subsubsection{oAW}
        
        \begin{itemize}
          \item Probleme w�hrend der Umsetzung?
          \item Erwartungshaltung?
          \item Umsetzung aus AndroMDA heraus
        \end{itemize}
        \section{Tests White/Blackbox?}
        \begin{itemize}
          \item Testf�lle aufstellen und diese mit JUnit umsetzen(nur f�r
          Android)
          \item Alternatives Testframework f�r iPhone \& Win7 Mobile?!?
        \end{itemize}